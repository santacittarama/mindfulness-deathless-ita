
\vspace*{0.3\onelineskip}
Scopo di questo libro e offrire istruzioni chiare e spunti di
riflessione circa la meditazione buddhista secondo l'insegnamento di
Ajahn Sumedho, un \textit{bhikkhu} (monaco) appartenente alla tradizione
\textit{Theravāda}. I capitoli che seguono sono estratti da discorsi
tenuti da Ajahn Sumedho a praticanti di meditazione per introdurli in
concreto alla saggezza buddhista. A tale saggezza si allude in genere
con il termine \textit{Dhamma}, ossia le cose ``così come sono''.

Vi consigliamo di utilizzare questo libretto come un manuale graduale.
Il primo capitolo e un'introduzione alla pratica meditativa in generale,
mentre i capitoli della seconda sezione possono essere letti in sequenza
e fatti seguire da un periodo di meditazione. Il terzo capitolo è una
riflessione sul tipo di comprensione che scaturisce dalla pratica.
Nell'ultimo si spiega come prendere i ``Rifugi e Precetti'', che collocano
la pratica meditativa nel più ampio contesto del tirocinio spirituale.
Rifugi e Precetti si possono richiedere formalmente a un membro del
\textit{Sangha} monastico oppure possono essere semplicemente oggetto di impegno
personale. Sono alla base dei mezzi tramite cui i valori spirituali sono
portati nel mondo.

La prima edizione del libro (2.000 copie) è uscita nel 1985, in
occasione dell'apertura del Centro buddhista di Amaravati, ed è stata
presto esaurita. Visto il favore incontrato, c'è stato chi si è offerto
di finanziarne una ristampa. A parte una più attenta correzione delle
bozze, il testo è rimasto invariato. Essendo il frutto di liberi
contributi e atti di servizio al \textit{Dhamma}, chiediamo ai lettori di
rispettare questa offerta e renderla disponibile gratuitamente.

Possano tutti gli esseri realizzare la Verità.

\bigskip
{\par\raggedleft
Venerable Sucitto,\\
Amaravati Buddhist Centre, maggio 1986\\
\par}

