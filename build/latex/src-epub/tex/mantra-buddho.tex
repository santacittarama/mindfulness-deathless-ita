
Nel caso di una mente molto attiva sul piano discorsivo può essere utile
ricorrere al mantra\footnote{Mantra: una parola con un significato religioso, la cui
ripetizione è uno strumento di meditazione.} \textit{Buddho}. Inspirate sul ``Bud'' ed espirate
sul ``dho'', concentrandovi su quest'unico pensiero per la durata
dell'atto respiratorio. È un modo per sostenere la concentrazione:
quindi, per i prossimi quindici minuti praticate \textit{ānāpānasati} mettendoci
tutta la vostra attenzione, raccogliendo la mente attorno alla parola
``\textit{Buddho}''. Imparate a portare la mente a quel livello di chiarezza e di
vivacità, invece di sprofondare in uno stato passivo. C'è bisogno di uno
sforzo sostenuto: un'inspirazione sul ``Bud'' – deve stagliarsi chiaro e
nitido nella mente, mantenendo in primo piano questo pensiero dal
principio alla fine dell'inspirazione – e l'espirazione sul ``dho''. In
questo momento non occupatevi d'altro. Ora è il momento di dedicarsi
solo a questo – i vostri problemi e i problemi del mondo potrete
risolverli più tardi. In questo momento la situazione non richiede
altro. Fate emergere il mantra alla coscienza. Sostenetelo
deliberatamente, che non sia una esperienza meccanica e ripetitiva,
buona solo a intontirvi.

Infondete energia alla mente, in modo tale che
l'inspirazione sul ``Bud'' sia un'inspirazione viva, non un ``Bud''
meccanico che si affievolisce subito perché non è mai ravvivato e
rinnovato dalla mente. Potete visualizzarlo come se fosse scritto, in
modo da concentrarvi pienamente sulla sillaba per tutta la durata
dell'inspirazione, dal principio alla fine. Poi ``dho'' sull'espirazione
segue la stessa modalità, in modo da dare una certa continuità allo
sforzo più che procedere per tentativi sporadici e intermittenti.

Notate l'eventuale presenza di pensieri ossessivi, di frasi senza senso
che occupano lo spazio mentale. Sprofondare in uno stato di passività
lascia mano libera ai pensieri ossessivi. Invece, imparare a capire come
funziona la mente e come adoperarla con intelligenza implica che si
scelga un determinato pensiero – l'idea di \textit{Buddho}, ossia il Buddha, il
Conoscitore – e che lo si sostenga in quanto pensiero. Non come un
pensiero abituale e ossessivo, ma come un'applicazione intelligente del
pensiero, allo scopo di sostenere la concentrazione per la durata di
un'inspirazione, di un'espirazione, per quindici minuti.

L'esercizio consiste in questo, tutte le volte che non ci si riesce e la
mente divaga, si prende semplicemente atto di essersi distratti, o che
si sta solo pensando di farlo, o che si preferirebbe lasciar perdere
``\textit{Buddho}'' (``non ho nessuna voglia di farlo; mi piacerebbe starmene seduto
a riposare senza sforzarmi di fare niente; proprio non mi va''). O forse
in questo momento c'è qualcos'altro che vi preoccupa, che cerca di
venire a galla nella mente conscia; in tal caso notate di che si tratta.
Notate lo stato d'animo predominante al momento, non per criticare o
scoraggiarvi, semplicemente notate con calma, spassionatamente, se
l'esercizio vi calma o vi rende opachi e sonnolenti, se non avete fatto
altro che pensare o vi siete concentrati. Fate ciò semplicemente per
saperlo.

L'ostacolo alla pratica della concentrazione è l'avversione per il
fallimento e un intenso desiderio di riuscire. Nella pratica non è in
gioco la forza di volontà, ma la saggezza, la saggezza del fare
attenzione. Questo esercizio vi darà modo di conoscere i vostri punti
deboli, dov'è che tendete a perdervi. Siete testimoni dei tratti del
carattere che avete sviluppato fino ad ora nella vostra vita, non per
criticarli ma solo per imparare a lavorarci sopra e non esserne schiavi.
Ciò implica un'accurata e saggia riflessione sulle cose così come sono.
Sicché, invece di evitarle a tutti i costi, anche le situazioni più
disperanti vengono osservate e riconosciute. È in gioco la capacita di
tollerare. Spesso il \textit{Nibbāna} viene descritto in termini di ``freddezza''.
Deprimente, eh? Ma la parola allude a qualcosa di preciso. Freddezza
rispetto a cosa?È qualcosa che dà sollievo, che non è travolto dalle
passioni ma distaccato, vigile ed equilibrato.

La parola ``\textit{Buddho}'' può essere coltivata nella vita come qualcosa con cui
riempire la mente al posto delle preoccupazioni e di ogni sorta di
abitudini malsane. Prendetela, osservatela, ascoltatela: \textit{Buddho}!
Significa ``colui che sa'', il Buddha, il risvegliato, la qualità
dell'essere svegli. La potete anche visualizzare. Ascoltate il
chiacchierio della mente – bla bla bla – che secerne senza sosta ogni
sorta di paure e avversioni represse. Quindi ne prendete coscienza. Non
usiamo \textit{Buddho} come una mazza con cui annientare e reprimere tutto, ma
come un abile espediente. Anche gli strumenti più raffinati si possono
usare per uccidere e danneggiare gli altri, no? Si può prendere una
pregevole statua del Buddha e spaccarla in testa a qualcuno, volendo!
Non è questa la \textit{Buddhanussati}, la riflessione sul Buddha, di cui
parliamo! Però è proprio così che potremmo usare la parola ``\textit{Buddho}'',
come un mezzo per reprimere pensieri e sentimenti. E sarebbe un uso
malsano. Ricordate che il nostro scopo non è sopprimere, ma lasciare che
le cose si dissolvano spontaneamente. È una pratica delicata in cui
pazientemente si sovrappone \textit{Buddho} ai pensieri, non per esasperazione,
ma in modo fermo e deliberato.

Il mondo ha proprio bisogno di imparare a farlo, vero? Gli Usa e
l'Unione Sovietica, avrebbero dovuto farlo invece di ricorrere alle
mitragliatrici e alle armi nucleari per annientare qualunque
opposizione... o invece di dirsi orribili cattiverie a vicenda. Lo
facciamo anche noi, non è vero? Quanti di voi hanno preso a male parole
qualcuno di recente, lo hanno ferito, criticato impietosamente, solo
perché vi irritava, vi ostacolava, o vi faceva paura? Allora cominciamo
a praticare con queste piccole esperienze cattive e irritanti nella
nostra mente, con le attività del pensiero che riteniamo futili e
sciocche. Usiamo \textit{Buddho} non come una mazza ma come un abile espediente
che permette ai fenomeni di passare, che li lascia andare. Ora, per i
prossimi quindici minuti tornate al vostro naso con la ripetizione del
mantra ``\textit{Buddho}''. Cercate di capire come funziona e come lavorarci.

