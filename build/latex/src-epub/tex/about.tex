
{\centering\par
\Large\scshape\chapTitleFont\thetitle
\par}
\vspace*{2\baselineskip}

{%
\setlength{\parskip}{1.5em}
\setlength{\parindent}{0pt}

Il materiale di questo libro è tratto da discorsi tenuti al monastero della Foresta di Chithurst (Regno Unito) nel Gennaio 1984, con l'eccezione dei capitoli ``\textit{Sforzo e rilassamento},'' ``\textit{Benevolenza}'' e ``\textit{Gli impedimenti e la cessazione degli impedimenti},'' che sono estratti da discorsi dati nel Dicembre del 1982 al monastero Internazionale della Foresta (Wat Pah Nanachat) di Ubon, nel nord-est della Thailandia.

% The photographs are of slabs from the ruins of the ancient \textit{stūpa} at Amaravati in Andhra Pradesh, India; they are reproduced by kind permission of the Trustees of the British Museum, London.
%
% Page \pageref{image-stupa}: the \textit{stūpa}, a monumental reliquary, contains the relics of a saint. As the object of pilgrimages, it symbolizes the universal quest for spiritual truth.
%
% Page \pageref{image-feet}: the iconographical footprints of the Buddha. They represent the path that a teacher has taken, to be followed by his disciples.
%
% Page \pageref{image-lotus-scene}: the lotus of wisdom in a scene of everyday human activity.
%
% Le fotografie mostrano lastre delle rovine dell'antico stupa di Amaravati in Andhra Pradesh, in India. Sono riprodotte su gentile concessione dell'amministrazione del British Museum di Londra.
%
% Pagina 22: lo stupa, un imponente reliquiario, contiene le reliquie di un santo. Come luogo di pellegrinaggio, simboleggia l'universale ricerca della verità spirituale.
%
% Pagina 36: le orme iconografiche del Buddha. Rappresentano il sentiero che l'insegnante ha intrapreso, e che sarà seguito dai discepoli.
%
% Pagina 100: il loto della saggezza in una scena quotidiana di vita umana.

\clearpage
\thispagestyle{empty}

{\scshape \theauthor} è nato negli USA, a Seattle nel 1934. Ha preso l'ordinazione come bhikkhu (monaco della tradizione Theravāda) in Thailandia nel 1966 e ha trascorso dieci anni nel nord-est del Paese con il Venerabile Ajahn Chah, insegnante della tradizione spirituale dei ``Maestri della Foresta.'' Invitato nel 1976 in Inghilterra da un'associazione laica (English Sangha Trust) ha successivamente fondato i monasteri di Chithurst e di Amaravati in Inghilterra, inoltre ha incoraggiato e sostenuto l'apertura di altri monasteri, fra cui il Santacittarama in Italia, contribuendo a diffondere la tradizione Theravāda in diversi paesi occidentali. Nel 2010, dopo 33 anni di servizio nelle comunità occidentali, ha deciso di lasciare i suoi incarichi pubblici e ritornare in Thailandia per continuare la sua vita monastica in forma più ritirata. Negli ultimi anni si è trasferito nuovamente in Inghilterra dove risiede al monastero di Amaravati.

{\scshape Ajahn Sucitto} è stato l'abate di Cittaviveka, monastero buddhista di Chithurst, dal 1992. Nel 1986 ha offerto il suo contributo a questo libretto, scrivendone la prefazione. È nato a Londra nel 1949. Nel Marzo 1976 venne ordinato bhikkhu in Thailandia. Ritornò in Gran Bretagna nel 1978 dove incominciò la sua formazione con Ajahn Sumedho al vihara buddhista di Hampstead, a Londra. Nel 1979 fece parte del piccolo gruppo storico di monaci che, con Ajahn Sumedho, fondò Cittaviveka, il monastero buddhista di Chithurst, nel West Sussex. Attualmente vive ancora a Cittaviveka, ma ha lasciato il ruolo di abate per dedicarsi maggiormente all'insegnamento.

}

