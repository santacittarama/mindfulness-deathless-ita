
La maggior parte delle istruzioni possono essere eseguite
indifferentemente nella posizione seduta, camminando o stando fermi in
piedi. Tuttavia, la consapevolezza del respiro (\textit{ānāpānasati}) di cui si
parla nei primi capitoli viene generalmente praticata nella posizione
seduta, in quanto si giova dell'associazione con uno stato fisico di
immobilità e calma. A tale scopo l'importante è sedersi tenendo la
colonna vertebrale eretta ma non tesa, con il collo allineato ad essa e
la testa bilanciata in modo che non ciondoli in avanti. Molti trovano
che la postura del loto (seduti a gambe incrociate su un cuscino o una
stuoia con uno o due piedi sulla coscia opposta e la pianta rivolta in
su) offre un equilibrio ideale fra stabilità e vigore, ovviamente dopo
alcuni mesi di pratica. È bene allenarsi ad assumere questa posizione
con gentilezza, un poco alla volta. Se risultasse troppo difficile, si
può usare una sedia con lo schienale diritto. Dopo aver trovato una
posizione equilibrata e stabile, rilassate le braccia e il volto,
lasciando che le mani riposino in grembo una sull'altra. Chiudete gli
occhi, rilassate la mente, rivolgete l'attenzione all'oggetto di
meditazione prescelto.

\label{jongrom}
\textit{Joṅgrom} è una parola in lingua thailandese che deriva dal
pali\footnote{Pali: è la lingua indiana in cui è redatta la versione
del Canone Buddhista della scuola \textit{Theravāda} (la Via degli Anziani).}
\textit{caṅkama}, che significa camminare avanti e indietro in
linea retta. Il tratto di strada, che misura idealmente dai venti ai
trenta passi, va scelto fra due oggetti chiaramente distinguibili, in
modo da non dover contare i passi. Le mani vanno unite ma non strette e
tenute davanti o dietro la schiena, con le braccia rilassate. Lo sguardo
va diretto in modo non focalizzato sul sentiero, su un punto a circa
dieci passi davanti a sé, senza osservare nulla in particolare ma solo
per mantenere un'inclinazione del collo il più possibile comoda. Quindi
si inizia a camminare con passo misurato e una volta arrivati in fondo
al tratto prescelto si resta fermi per la durata di uno o due respiri,
ci si volta consapevolmente e consapevolmente si torna indietro.

